\documentclass[11pt]{article}
\textwidth 7.5in
\textheight 10in
\oddsidemargin -.5in
\evensidemargin -.5in
\topmargin -1in    
\usepackage{graphics}        
\usepackage{graphicx}
\usepackage{amsmath,amsthm,amscd,amssymb}
\usepackage{latexsym}
\usepackage{upref}
\begin{document}
\newcommand{\dsp}{\displaystyle}
\newcommand{\ihat}{{\bf{i}}}
\newcommand{\jhat}{{\bf{j}}}
\newcommand{\khat}{{\bf{k}}}
\newcommand{\Fhat}{{\bf{F}}}
%\pagestyle{headings}
%\markright{MATH 1120 Lab Worksheet \#1}

\thispagestyle{empty}

\noindent
\sffamily
\begin{center}
\rule{7.5in}{2pt}

\vspace{.2in}

\begin{tabular}{p{4in}p{3.5in}}
MATH 2825

\vspace{.2in}

Programming Lab \#1
& 
NAME:  $_{\rule{2.5in}{1pt}}$

\vspace{.2in}

Due Tuesday February 5th 2019 by 11:59 P.M.
\\*[.2in]
\end{tabular}
\rule{7.5in}{2pt}

\vspace{.1in}
All work is to be done in a programming notebook either Mathematica and Jupyter (python3).  \\
 Please refer to the blackboard site for commands and examples. \\
 Submissions must be made electronically on blackboard, consider using a GitHub repository to store your code. \\
   You will be graded on the output that I am able to generate from your commands.
\end{center}
%\begin{tabular}{*{26}{|c}|}\hline
%A&B&C&D&E&F&G&H&I&J&K&L&M&N&O&P&Q&R&S&T&U&V&W&X&Y&Z\\ \hline
%1&2&3&4&5&6&7&8&9&10&11&12&13&14&15&16&17&18&19&20&21&22&23&24&25&26\\ \hline
%\end{tabular}

\begin{enumerate}
\item Let the last digit of your student number be $a$ (if it happens to be a zero use $a=10$).  Consider the function 
\[f(x)=\sqrt{ax+1}.\]
\begin{enumerate}
%\item Graph the function from 0 to 1.
\item Define the function $f(x)$ in the programming language.  Compute $f(0)$.
\item Find the value $b$ for which the domain of the function is $[b,\infty)$.  Be sure to explain in words how you found $b$.
\item Plot the function from $b$ to 5.

\end{enumerate}
\item Consider the function built by compositions, where $a$ is as defined above and $n$ is a natural number,
\[
f_0(x)=\frac1{a-x}\quad\quad\text{and}\quad\quad f_{n+1}=f_0\circ f_n.
\]
\begin{enumerate}
\item Plot $f_0,\ f_1,\ f_2,\ f_3$ on the same screen and describe the effects of repeated composition.
\item Make a prediction for what function $f_n(x)$ might be.  Explain in words. 
\end{enumerate}
\item The point $P(1,0)$ lies on the curve $y=\sin\left(\frac{2a\pi}x\right)$, for $a$ as defined above.
\begin{enumerate}
\item If $Q$ is the point $\left(x,\sin\left(\frac{2a\pi}x\right)\right)$, find the slope of the secant line $PQ$ for $x=2$, $x=1.5$, $x=1.1$, $x=1.01$ and $x=1.001$  You can do each individually or use an automation in the programming language.  Do these slopes appear to be approaching a limit, explain in words. \label{a}
\item Use a graph of the curve to explain why the slopes of the secant lines in part \ref{a} are not close to the slopes of the tangent line at $P$.
\item By choosing appropriate secant lines, estimate the slope of the tangent line at $P$.
\end{enumerate}
\end{enumerate}
\end{document}





