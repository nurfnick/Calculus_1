\documentclass[11pt]{article}
\textwidth 7.5in
\textheight 10in
\oddsidemargin -.5in
\evensidemargin -.5in
\topmargin -1in    
\usepackage{graphics}        
\usepackage{graphicx}
\usepackage{amsmath,amsthm,amscd,amssymb}
\usepackage{latexsym}
\usepackage{upref}
\usepackage{hyperref}
\begin{document}
\newcommand{\dsp}{\displaystyle}
\newcommand{\ihat}{{\bf{i}}}
\newcommand{\jhat}{{\bf{j}}}
\newcommand{\khat}{{\bf{k}}}
\newcommand{\Fhat}{{\bf{F}}}
%\pagestyle{headings}
%\markright{MATH 1120 Lab Worksheet \#1}

\thispagestyle{empty}

\noindent
\sffamily
\begin{center}
\rule{7.5in}{2pt}

\vspace{.2in}

\begin{tabular}{p{4in}p{3.5in}}
MATH 2825

\vspace{.2in}

Mathematica Lab \#2
& 
NAME:  $_{\rule{2.5in}{1pt}}$

\vspace{.2in}

Due Thursday October 6th 2022 by 11:59 P.M.
\\*[.2in]
\end{tabular}
\rule{7.5in}{2pt}

\vspace{.1in}
All work is to be done in a mathematica or Jupyter notebook. \\
 Please refer to the \href{https://github.com/nurfnick/Calculus_1}{GitHub Repo} for commands and examples. \\
 Submissions must be made electronically to blackboard \\
   You will be graded on the output that I am able to generate from your commands.

\end{center}
\begin{tabular}{*{26}{|c}|}\hline
A&B&C&D&E&F&G&H&I&J&K&L&M&N&O&P&Q&R&S&T&U&V&W&X&Y&Z\\ \hline
1&2&3&4&5&6&7&8&9&10&11&12&13&14&15&16&17&18&19&20&21&22&23&24&25&26\\ \hline
\end{tabular}

\begin{enumerate}
\item Let $a$ be the digit corresponding to your first initial in the table above.  Consider the function 
\[f(x)=\frac{\tan (ax)}x.\]
\begin{enumerate}
%\item Graph the function from 0 to 1.
\item Graph the function on a small enough interval to identify what is happening when $x=0$.
\item In words, guess what the $\lim_{x\to0}f(x)$ is equal to.
\item Use the computer to compute the limit.
\item Do you think this is always the value or is somehow specific to your problem?  Explain in words.
\end{enumerate}




\item Consider the limit, where $a$ is as defined above,
\[
\lim_{x\to0}\frac{e^{ax}-1}{x}.
\]
\begin{enumerate}
\item Find the limit.
\item Use the precise definition of the limit to find the value of $\delta$ that corresponds to an $\epsilon=.5$. Explain in words how you did this.

\end{enumerate}
\item Consider the function, where $a$ is as defined above
\[
g(x)=6x+a\sin x
\]
\begin{enumerate}
\item Find the derivative of the function.
\item Find the equation of the tangent line at the point $\left(\frac\pi6,\pi+\frac a2\right)$.
\item Graph both the function and the line on the same axis.
\end{enumerate}
\end{enumerate}
\end{document}





