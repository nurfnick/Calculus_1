\documentclass[11pt]{article}
\textwidth 7.5in
\textheight 10in
\oddsidemargin -.5in
\evensidemargin -.5in
\topmargin -1in    
\usepackage{graphics}        
\usepackage{graphicx}
\usepackage{amsmath,amsthm,amscd,amssymb}
\usepackage{latexsym}
\usepackage{upref}
\begin{document}
\newcommand{\dsp}{\displaystyle}
\newcommand{\ihat}{{\bf{i}}}
\newcommand{\jhat}{{\bf{j}}}
\newcommand{\khat}{{\bf{k}}}
\newcommand{\Fhat}{{\bf{F}}}
%\pagestyle{headings}
%\markright{MATH 1120 Lab Worksheet \#1}

\thispagestyle{empty}

\noindent
\sffamily
\begin{center}
\rule{7.5in}{2pt}

\vspace{.2in}

\begin{tabular}{p{4in}p{3.5in}}
MATH 2825

\vspace{.2in}

Mathematica Lab \#4
& 
NAME:  $_{\rule{2.5in}{1pt}}$

\vspace{.2in}

Due Monday December 5th 2022 by 11:30 A.M.
\\*[.2in]
\end{tabular}
\rule{7.5in}{2pt}

\vspace{.1in}
All work is to be done in a programming notebook either Mathematica and Jupyter (python3).  \\
 Please refer to the blackboard site for commands and examples. \\
 Submissions must be made electronically on blackboard, consider using a GitHub repository to store your code. \\
   You will be graded on the output that I am able to generate from your commands.
\end{center}
%\begin{tabular}{*{26}{|c}|}\hline
%A&B&C&D&E&F&G&H&I&J&K&L&M&N&O&P&Q&R&S&T&U&V&W&X&Y&Z\\ \hline
%1&2&3&4&5&6&7&8&9&10&11&12&13&14&15&16&17&18&19&20&21&22&23&24&25&26\\ \hline
%\end{tabular}

\begin{enumerate}
\item Let $a$ be the digit corresponding to the last digit of your student id number, if it is 0 use 10.  Consider the function 
\[f(x)=x^6-x^5- a x^4-x^2+x-1\]
\begin{enumerate}
\item Graph the function on an interval so that you can approximate the roots of the function.  State your approximations clearly.
\item Use a built-in command to find all roots.  These built-in command may use Newton's method to solve for roots.
\end{enumerate}




\item Consider the curve $y=x^a$, where $a$ is defined as above except if 1 use 11.

\begin{enumerate}
\item Graph the equation from 0 to 2. 
\item Express the area under the curve as a limit and compute this limit.
\item Check your answer by using a definite integral.  Explain in words why these must be the same.

\end{enumerate}
\item Consider the function, 
\[
Si(x)=\int_0^x\frac{\sin t}t dt
\]
\begin{enumerate}
\item Plot $Si(x)$ for $0\leq x\leq 20$.
\item Find the values for $x$ where there are local extrema, explain in words where all values would be.
\item Solve $Si(x)=1$.
\end{enumerate}
\end{enumerate}
\end{document}





